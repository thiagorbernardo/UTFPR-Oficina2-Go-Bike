\chapter{Introdução}
Este projeto visa, em suma, trazer uma ampla gama de funcionalidades e praticidades para o ciclista, com baixo custo e sem que este necessite de seu celular durante a atividade.
\section{Motivação} %qual o problema?
Quando um ciclista sai para pedalar, muitas vezes não é prático e é até perigoso que ele leve o seu celular no bolso ou em um suporte na bicicleta, visto que pode derrubar o aparelho, se acidentar ou até mesmo ser roubado em um momento de distração ou descuido.
Tendo em mente esses problemas a ideia do GoBike foi concebida com o intuito de resolver esse problema, sendo um módulo que pode ser removível da bicicleta que será capaz de captar a velocidade e localização do usuário, e posteriormente trazer todas essas informações em um aplicativo.
\section{Objetivos}
\subsection{Objetivo geral} 
O objetivo da realização deste projeto é desenvolver um sistema integrado à bicicleta que possibilite acesso à um conjunto amplo de funções úteis para o ciclista, como a velocidade e compartilhamento de localização, sem que este precise carregar consigo o seu celular durante a atividade de pedalar, assim tornando a prática mais segura trazendo conforto para os amigos e familiares. 
\subsection{Objetivos específicos}
Construção de um dispositivo que consiga:
\begin{itemize}
  \item Ser removível da bicicleta.
  \item Verificar a velocidade através de uma tela na bicicleta.
  \item Rastrear a localização do módulo.
  \item Estacionar a bicicleta e entrar em modo de segurança.
  \item Enviar notificações para o aplicativo para avisar quando a bicicleta está saindo do modo de segurança.
  \item Enviar notificações para o aplicativo caso a bicicleta esteja saindo do lugar e avisar de possíveis roubos.
  \item Permitir o compartilhamento da localização do módulo com amigos e familiares através do aplicativo.
  \item Obter as direções do módulo através da última localização disponível.
\end{itemize}

\subsection{Requisitos}

\textbf {Requisitos Funcionais (RF)}
\begin{itemize}
  \item RF01 - O aplicativo deve permitir o rastreio da bicicleta do usuário, ser capaz de ativar o modo de estacionar, receber notificações quando em modo estacionar e ver a localização de bicicleta de amigos.
  \item RF02 - O equipamento deve ser capaz de medir a velocidade do usuário.
  \item RF03 - O software deve salvar a posição do usuário.
  \item RF04 - O software deve se conectar ao celular via bluetooth.
  \item RF05 - O sistema deve possuir conexão com a internet via chip (SIM).
  \item RF06 - O hardware deve possuir um display para visualizar as informações.
  \item RF07 - A tela deve exibir a temperatura, atualizada de tempos em tempos.
  \item RF08 - A parte mecânica deve estar acoplada na bicicleta, estrutura demonstrada na Figura 4.
  \item RF09 - O software deve registrar as informações em um banco de dados.
  \item RF10 - O sistema deve possuir um botão para estacionar a bicicleta.
\end{itemize}
\newpage
\textbf {Requisitos Não-Funcionais (RNF)}
\begin{itemize}
  \item RNF01 - O app deve ser compatível com Android.
  \item RNF02 - A bateria deve durar 20 minutos.
  \item RNF03 - A parte mecânica deve permanecer estável enquanto o usuário pedala.
  \item RNF04 - O sistema deve ser removível.
  \item RNF05 - O sistema deve minimizar o consumo de internet.
\end{itemize}
