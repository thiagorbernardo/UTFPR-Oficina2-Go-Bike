\chapter{Conclusões}
Sendo este um projeto com duração de um semestre, em muitos momentos o projeto estava se desenvolvendo bem, e em muitos outros não tão bem. Todo o processo de design thinking, desde a idealização até a concretização do projeto, foi feito em congruência com as demais disciplinas e demandas da faculdade, o que requeriu uma disciplina, foco e comprometimento ainda maior de toda equipe. 

\section{Conclusões}
Com o desenvolvimento deste projeto, foi possível obter diversas conclusões e aprendizados valiosos. Foram desenvolvidos e engrandecidos conhecimentos relacionados a desenvolvimento de projetos, eletrônica, programação, cronograma, soldagem, trabalho em equipe e muitos outros mais. 

Do ponto de vista do desenvolvimento, foi possível visualizar a real importância de um bom planejamento inicial, cronograma e definição do escopo do projeto. Em diversos momentos foi necessário um plano B ou até C, e a tomada de decisões rápida, diálogo e sintonia da equipe foram essenciais nestes momentos.

Por fim, sentiu-se que os pontos mais críticos do projeto foram o desenvolvimento e montagem do hardware e a integração de todas as partes. Muitas vezes, módulos que funcionavam separados não funcionaram em conjunto, e componentes paravam de funcionar de um dia para o outro, ocorridos estes que tornaram-se os maiores pontos de dificuldade do nosso projeto. Saber lidar com imprevistos, ter paciência e tomar decisões planejadas reforçaram a confiança e motivação da equipe em diversos momentos no decorrer do semestre.

\section{Trabalhos futuros}
Para futuros projetos, com certeza serão levados ainda mais em consideração os possíveis erros e problemas com hardware e integração, prevendo tempo de trabalho e opções paralelas para que estes não sejam gargalos no desenvolvimento. 

Além disso, serão levados todos os aprendizados e conhecimentos para futuros projetos, tanto profissionais quanto acadêmicos, pois a abordagem positiva para obter soluções é a melhor técnica quando se pretende desenvolver algum projeto, e neste tivemos oportunidade de exercitar e aprimorar constantemente estes conhecimentos e muitos outros mais.

%ex.: melhorias,  entre outros.